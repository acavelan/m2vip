\documentclass[12pt,a4paper,utf8x]{report}
\usepackage[utf8x]{inputenc}
\usepackage[T1]{fontenc}
\usepackage[frenchb]{babel}

% Pour pouvoir utiliser
\usepackage{ucs}

\usepackage{url} % Pour avoir de belles url
\usepackage {geometry}

% Pour mettre du code source
\usepackage {listings}
% Pour pouvoir passer en paysage
\usepackage{lscape}

% Pour pouvoir faire plusieurs colonnes
\usepackage {multicol}

% Pour crééer un index
\usepackage{makeidx}
\usepackage{calc} 
% Pour la couleur
\usepackage{color}

% Pour les images
\usepackage{graphicx}
\usepackage{float} % Pour placer les image là on lui dit de les placer

\usepackage[pdftex,pdfborder={0 0 0}]{hyperref}

\usepackage{tikz} % Explication : http://d.xav.free.fr/tikz/index.html
\usepackage{textcomp}
\usepackage{pgfplots} % Plus simple


\begin{document}


\begin{figure}[H]
    \center
        \begin{tikzpicture}
                       \begin{axis}[ xlabel=Nombre de tranches utilis\'{e}s, ylabel=Temps d'exécution (heures), grid=major, xmin=1, xmax=48, ymin=0, width=12cm, height=7.5cm, xtick={1, 2, 4, 8, 16, 24, 48}]
                \addplot[color=blue,mark=x] coordinates
                {
                    (1,  22.9)
                    (2,   13.28)
                    (4,  8.58)
                    (8,  4.5)
                    (16, 2.28)
                    (24,  1.61 )
                    (48,  0.88)
                };
            \end{axis}
        \end{tikzpicture}
    \caption{Temps d'exécution du calcul }
\end{figure}

\end{document}
